\documentclass[swedish]{article}

\author{Mathias Magnusson}
\title{Le gymnasiearbete}
\date{}

\setcounter{tocdepth}{2}
\renewcommand*\contentsname{Innehållsförteckning}

\usepackage[
	pdfborder={0 0 0},
	colorlinks=true,
	linkcolor=black,
	urlcolor=blue
]{hyperref}

\begin{document}

\maketitle{}

\section*{Abstract}

Lets be very abstract about things here and speak the little English.

\clearpage

\tableofcontents

\clearpage

\section{Inledning}

\subsection{Syfte}

Syftet med detta gymnasiearbete är att skapa en webbsida där användaren ska
kunna öva på programmeringsproblemlösning och därmed få en förståelse för hur
en säker webbserver skapas samt hur opålitlig kod kan exekveras på ett
(relativt) säkert sätt.

\subsection{Bakgrund}

\subsection{Frågeställningar}

\begin{itemize}
	\item Hur exekveras opålitlig kod på ett säkert sätt? Alltså utan att en
		illvillig användare har möjligheten att:
		\begin{itemize}
			\item Läcka de specifika testdata som körs för att avgöra huruvida
				användaren har klarat uppgiften.
			\item Komma åt känslig användardata.
			\item På övrigt sätt utnyttja servern där webbsidan hostas.
		\end{itemize}
	\item
		Hur mäts ett programs körningstid samt minnesallokeringsmängd på ett
		konsekvent sätt?
	\item
		Hur implementeras en både tillfredsställande och säker
		användarupplevelse?

\end{itemize}

% \subsection{Metod}

\subsection{Material}

Mina bara händer och BIG BRAIN.

\subsection{Avgränsningar}

\subsection{Termer}

\begin{itemize}
	\item Server(?)
\end{itemize}

\section{Genomförande}

\subsection{Planering}

Projektets genomförande inleddes med att undersöka hur opålitlig kod, inlämnat
av användare, kan exekveras utan att servern utsätts för allvarliga
säkerhetsrisker. I början var tanken att köra programmen i en Docker-instans.
\href{https://www.docker.com}{Docker} har förmågan att skapa s.k.
\textit{containrar}, vilka fungerar likt egna datorer eller virtuella maskiner,
då de har sitt egna filsystem o.dyl.

Planen var att skapa en ny container varje gång en användare lämnar in ett
program för körning. Docker är inte gjort att användas så och några problem som
skulle uppstå är:
\begin{itemize}
	\item Prestandakravet skulle snabbt bli väldigt högt, då ${prestandan}^{-1}$
		som krävs för att starta en Docker-container är betydligt högre än det
		som krävs för att start ett vanligt program.
	\item Om något skulle gå fel med webbservern när en användares program körs
		finns risken att vissa containrar aldrig skulle bli avstängda och
		borttagna.
	\item Om programmet har tillgång till hela dess container är det näst
		intill omöjligt att hindra det från att komma åt internet och därmed
		möjligheten att bland annat läcka testdata.
\end{itemize}
Med detta i åtanke lämnades idén om att använda Docker.

% TODO: g&b=?=retarded
I grund och botten är det två huvudområden där användarens program måste
isoleras: åtkomst till filsystemet, samt åtkomst till internet. Om båda dessa är
helt förhindrade är det enda ett program kan göra att läsa in data från dess
inmatningsström, skriva ut data till dess utmatningsström, samt utföra
beräkningar på denna data. Inget av detta utgör säkerhetsrisker.

\subsection{Isolering från filsystem}

För att förhindra användarens program från att komma åt filer på servern används
Linux-syscallet \texttt{chroot}. Chroot står för change root,
\textit{ändra rot}, och dess effekt är att "på låtsas" ändra filsystemets rot
för programmet som körs. På så sätt kan ett programs tillgång till filsystemet
begränsas till en viss mapp, vari hur stor eller liten del av det övriga
filsystemets som än önskas kan tillgängliggöras. Den stora fördelen med denna
flexibilitet är att åtkomst till flertalet filer kan krävas för att köra vissa
program. För att exempelvis köra ett program skrivet i Java krävs bland annat en
\textit{Java virtual machine}, samt ytterligare några bibliotek som
\textit{JVM}en laddar in när den körs.

Man måste vara root.

setuid-bit

suid(annan uid)

C-skriptet

\section{Resultat}

\section{Diskussion / Slutsatser}

\section{Referenser}

\section{Bilagor}

\end{document}
