\documentclass{article}

\author{Mathias Magnusson}
\title{Le gymnasiearbete}
\date{}

\setcounter{tocdepth}{2}
\renewcommand*\contentsname{Innehållsförteckning}

\begin{document}

\maketitle{}

\section*{Abstract}

Lets be very abstract about things here and speak the little English.

\clearpage

\tableofcontents

\clearpage

\section{Inledning}

\subsection{Syfte}

Syftet med detta gymnasiearbete är att skapa en webbsida där användaren ska
kunna öva på programmeringsproblemlösning och därmed få en förståelse för hur
en säker webbserver skapas samt hur användarinlämnad kod exekveras på ett
(relativt) säkert sätt.

\subsection{Bakgrund}

\subsection{Frågeställningar}

\begin{itemize}
	\item Hur exekveras användarinlämnad kod på ett säkert sätt?
	\item
		Hur mäts ett programs körningstid samt minnesallokeringsmängd på ett
		konsekvent sätt?
	\item b

\end{itemize}

\subsubsection{}

\subsection{Metod}

\subsection{Material}

\subsection{Avgränsningar}

\section{Genomförande}

\section{Resultat}

\section{Diskussion / Slutsatser}

\section{Referenser}

\section{Bilagor}

\end{document}
