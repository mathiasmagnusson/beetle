\documentclass{article}

\author{Mathias Magnusson}
\title{Le gymnasiearbete}
\date{}

\setcounter{tocdepth}{2}
\renewcommand*\contentsname{Innehållsförteckning}

\usepackage[
	pdfborder={0 0 0},
	colorlinks=true,
	linkcolor=black,
	urlcolor=blue
]{hyperref}

\begin{document}

\maketitle{}

\section*{Abstract}

Lets be very abstract about things here and speak the little English.

\clearpage

\tableofcontents

\clearpage

\section{Inledning}

\subsection{Syfte}

Syftet med detta gymnasiearbete är att skapa en webbsida där användaren ska
kunna öva på programmeringsproblemlösning och därmed få en förståelse för hur
en säker webbserver skapas samt hur opålitlig kod kan exekveras på ett
(relativt) säkert sätt.

\subsection{Bakgrund}

\subsection{Frågeställningar}

\begin{itemize}
	\item Hur exekveras opålitlig kod på ett säkert sätt?
	\item
		Hur mäts ett programs körningstid samt minnesallokeringsmängd på ett
		konsekvent sätt?
	\item
		Hur implementeras en både tillfredsställande och säker
		användarupplevelse?

\end{itemize}

% \subsection{Metod}

\subsection{Material}

Mina bara händer och BIG BRAIN.

\subsection{Avgränsningar}

\section{Genomförande}

Projektets genomförande inleddes med att undersöka hur opålitlig kod, inlämnat
av användare, kan exekveras utan att servern utsätts för större säkerhetsrisker.
I början var tanken om att köra programmen i en docker-instans.
\href{https://www.docker.com}{Docker} är som en Virtuell

\section{Resultat}

\section{Diskussion / Slutsatser}

\section{Referenser}

\section{Bilagor}

\end{document}
